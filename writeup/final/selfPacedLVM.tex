\documentclass{article}
\usepackage{nips10submit_e,times}
\usepackage{amssymb}
\usepackage{amsmath}
\usepackage{epsfig}
\usepackage{wrapfig}
\usepackage[numbers]{natbib}
\usepackage{algorithm}
\usepackage{algorithmic}

\renewcommand{\baselinestretch}{.99}
  
\newcommand{\comment}[1]{}
\newcommand{\mysection}[1]{\vspace{-4mm}\section{#1}\vspace{-4mm}}
\newcommand{\mysubsection}[1]{\vspace{-3mm}\subsection{#1}\vspace{-3mm}}
\newcommand{\myparagraph}[1]{\vspace{-2mm}\paragraph{#1}}
\newcommand{\mycaption}[1]{\vspace{-3mm}\caption{\em \footnotesize #1}\vspace{-3mm}}
\newcommand{\mytopcaption}[1]{\caption{\em \footnotesize #1}}
\newcommand{\theHalgorithm}{\arabic{algorithm}}
\newcommand{\twofigures}[5]{
\renewcommand{\citename}{\citet}
\begin{minipage}[t]{0.49\textwidth}
\centerline{\psfig{figure=#1,height=#3}}
\end{minipage}\hspace{0.01\textwidth}
\begin{minipage}[t]{0.49\textwidth}
\centerline{\psfig{figure=#2,height=#3}}
\end{minipage}
\raisebox{0mm}
{\makebox[0.5\textwidth][c]{(#4)}\makebox[0.5\textwidth][c]{(#5)}}
}
\newcommand{\threefigures}[4]{
\begin{minipage}[t]{0.32\textwidth}
\centerline{\psfig{figure=#1,height=#4}}
\end{minipage}\hspace{0.01\textwidth}
\begin{minipage}[t]{0.32\textwidth}
\centerline{\psfig{figure=#2,height=#4}}
\end{minipage}\hspace{0.01\textwidth}
\begin{minipage}[t]{0.32\textwidth}
\centerline{\psfig{figure=#3,height=#4}}
\end{minipage}
}
\newcommand{\threefigurescaps}[7]{
\begin{minipage}[t]{0.32\textwidth}
\centerline{\psfig{figure=#1,height=#4}}
\end{minipage}\hspace{0.01\textwidth}
\begin{minipage}[t]{0.32\textwidth}
\centerline{\psfig{figure=#2,height=#4}}
\end{minipage}\hspace{0.01\textwidth}
\begin{minipage}[t]{0.32\textwidth}
\centerline{\psfig{figure=#3,height=#4}}
\end{minipage}
\raisebox{0mm}
{\makebox[0.32\textwidth][c]{(#5)}\makebox[0.32\textwidth][c]{(#6)}
\makebox[0.32\textwidth][c]{(#7)}
}
}
\newcommand{\argmin}{\operatornamewithlimits{argmin}}
\newcommand{\argmax}{\operatornamewithlimits{argmax}}

\title{Self-Paced Learning for Latent Variable Models}

\author{
M.\ Pawan Kumar ~~~~~ Benjamin Packer ~~~~~ Daphne Koller\\
Computer Science Department \\
Stanford University \\
\texttt{\{pawan,bpacker,koller\}@cs.stanford.edu}
}

\nipsfinalcopy

\begin{document} 

\maketitle
\vspace{-8mm}

\begin{abstract}
\vspace{-4mm}
Latent variable models are a powerful tool for addressing several tasks
in machine learning. However, the algorithms for learning the parameters
of latent variable models are prone to getting stuck in a bad local optimum.
To alleviate this problem, we build on the intuition
that, rather than considering all samples simultaneously, the algorithm should be presented with the
training data in a {\em meaningful} order that facilitates learning. 
The order of the samples is determined by how {\em easy} they are. The main challenge
is that often we are not provided with
a readily computable measure of the easiness of samples. We address this issue by 
proposing a novel, iterative {\em self-paced learning} algorithm where each iteration
simultaneously selects easy samples and learns a new parameter vector.
The number of samples selected is governed by
a weight that is annealed until the entire training data has been considered.
We empirically demonstrate that the
self-paced learning algorithm outperforms the state of the art method for
learning a latent structural {\sc svm} on
four applications: object localization, noun phrase coreference, motif finding and handwritten digit recognition.
\end{abstract}
\vspace{-4mm}

\mysection{Introduction}
\label{sec:introduction}

Latent variable models provide an elegant formulation for several applications
of machine learning. For example, in
computer vision, we may have many `car' images from which
we wish to learn a `car' model. However, the exact location
of the cars may be unknown and can be modeled as latent
variables. In medical diagnosis, learning to diagnose a
disease based on symptoms can be improved by treating unknown or
unobserved diseases as latent variables (to deal with confounding
factors). Learning the parameters of a latent variable model often
requires solving a non-convex optimization problem. Some common approaches
for obtaining an approximate solution include the well-known {\sc em}~\cite{dempsterjrss77}
and {\sc cccp} algorithms~\cite{felzenszwalbcvpr08,yuicml09,yuillenc03}.
However, these approaches are prone to getting stuck in a bad local minimum with
high training and generalization error.

Machine learning literature is filled with scenarios in which one is
required to solve a non-convex optimization task, for example learning
perceptrons or deep belief nets. A common approach for avoiding a bad
local minimum in these cases is to use multiple runs with random
initializations and pick the best solution amongst them (as
determined, for example, by testing on a validation set). However,
this approach is adhoc and computationally expensive as one may be required to
use several runs to obtain an accurate solution.
Bengio {\em et al.}~\cite{bengioicml09} recently proposed an
alternative method for training with non-convex objectives, called
curriculum learning. The idea is inspired by the way children are
taught: start with easier concepts (for example, recognizing objects
in simple scenes where an object is clearly visible) and build up to
more complex ones (for example, cluttered images with occlusions).
Curriculum learning suggests using the easy samples first and
gradually introducing the learning algorithm to more complex ones.
The main challenge in using the curriculum learning strategy is that
it requires the identification of easy and hard samples in a given
training dataset.  However, in many real-world applications, such a
ranking of training samples may be onerous or conceptually difficult
for a human to provide --- even if this additional human supervision
can be provided, what is intuitively ``easy'' for a human may not
match what is easy for the algorithm in the feature and hypothesis
space employed for the given application.

To alleviate these deficiencies, we introduce {\em self-paced
learning}. In the context of human education, self-paced learning
refers to a system where the curriculum is determined by the pupil's
abilities rather than being fixed by a teacher. We build on this
intuition for learning latent variable models by designing an
iterative approach that simultaneously selects easy samples and
updates the parameters at each iteration. The number of samples
selected at each iteration is determined by a weight that is gradually
annealed such that later iterations introduce more samples.  The
algorithm converges when all samples have been considered and the
objective function cannot be improved further.  Note that, in
self-paced learning, the characterization of what is ``easy'' applies
not to individual samples, but to sets of samples; {\em a set of
samples is easy if it admits a good fit in the model space}.

We empirically demonstrate that
our self-paced learning approach outperforms the state of the art algorithm for
learning a recently proposed latent variable model, called latent structural
{\sc svm}, on four standard machine learning applications using publicly
available datasets.

\mysection{Related Work}
\label{sec:relatedwork}
Self-paced learning is related to curriculum learning in that both regimes
suggest processing the samples in a meaningful order.
Bengio {\em et al.\ }\cite{bengioicml09} noted that curriculum learning can be seen as
a type of continuation method~\cite{allgower90}. However, in their work, they circumvented the
challenge of obtaining such an ordering by using datasets where there is a clear distinction
between easy and hard samples (for example, classifying equilateral triangles vs.\ squares is easier than
classifying general triangles vs.\ general quadrilaterals). Such datasets are rarely available in
real world applications, so it is not surprising that the experiments in~\cite{bengioicml09} were mostly
restricted to small toy examples.

Our approach also has a similar flavor to active learning, which
chooses a sample to learn from at each iteration. Active learning approaches differ in their sample selection criteria.
For example, Tong and Koller~\cite{tongjmlr01} suggest choosing a sample that is close to the margin (a ``hard'' sample),
corresponding to {\em anti-curriculum learning}. Cohn {\em et al.\ }\cite{cohnjair96} advocate the use of the 
most uncertain sample with respect to the current classifier. However, unlike our setting, in active learning the labels of all the samples are not known when the samples are
chosen.

Another related learning regime is co-training, which works by alternately training classifiers such that
the most confidently labeled samples from one classifier are used to train the other~\cite{blumcolt98,nigamcikm00}.
Our approach differs from co-training in that in our setting the latent variables are simply used to assist in predicting
the target labels, which are always observed, whereas co-training deals with a semi-supervised setting in which some labels
are missing. 

\mysection{Preliminaries}
\label{sec:preliminaries}

We will denote the training data as ${\cal D} = \{({\bf x}_i,{\bf y}_i),\cdots,({\bf x}_n,{\bf y}_n)\}$, where ${\bf x}_i \in {\cal X}$
are the observed variables (which we refer to as input) for the $i^{th}$ sample
and ${\bf y}_i \in {\cal Y}$ are the unobserved variables (which we refer to as output),
whose values are known during training. In addition, latent variable models also contain latent, or hidden, variables that we
denote by ${\bf h}_i \in {\cal H}$.
For example, when learning a `car' model using image-level labels, {\bf x} represents an image, the binary output {\bf y} indicates the presence or
absence of a car in the image, and {\bf h} represents the car's bounding box (if present).

Given the training data, the parameters {\bf w} of a latent variable model are learned by optimizing
an objective function, for example by maximizing the likelihood of ${\cal D}$ or minimizing the risk over ${\cal D}$. Typically,
the learning algorithm proceeds iteratively, with each iteration consisting of two stages: (i) the hidden variables are either imputed
or marginalized to obtain an estimate of the objective function that only depends on {\bf w};
and (ii) the estimate of the objective function is optimized to obtain a new set of parameters. We briefly describe two such well-known
algorithms below.

\paragraph{EM Algorithm for Likelihood Maximization.} An intuitive objective is to maximize likelihood:
%An intuitive objective for learning the parameters {\bf w} would be to maximize the likelihood
%of the training data ${\cal D}$, that is,
\begin{equation}
\max_{\bf w} \sum_i \log \Pr({\bf x}_i, {\bf y}_i; {\bf w}) = 
\max_{\bf w} \left( \sum_i \log \Pr({\bf x}_i, {\bf y}_i, {\bf h}_i; {\bf w}) - 
\sum_i \log \Pr({\bf h}_i | {\bf x}_i, {\bf y}_i; {\bf w})\right).
\label{eq:maximumLikelihood}
\end{equation}
A common approach for this task is to use the {\sc
em} method~\cite{dempsterjrss77} or one of its many
variants~\cite{gelman95}.  Outlined in Algorithm~\ref{algo:EM}, {\sc
em} iterates between finding the expected value of the latent variables
{\bf h} and maximizing objective ~(\ref{eq:maximumLikelihood})
subject to this expectation.
%The key observation in {\sc em} is that at iteration $t$ the second term in the {\sc rhs} of objective~(\ref{eq:maximumLikelihood}) ---
%$\sum_i \log \Pr({\bf h}_i | {\bf x}_i, {\bf y}_i; {\bf w})$ --- is maximized when ${\bf w} = {\bf w}_t$. Therefore the new parameters
%${\bf w}_{t+1}$, which increase the likelihood, can be obtained by maximizing only the first term of 
%objective~(\ref{eq:maximumLikelihood}) as shown in step 4 of
%Algorithm~\ref{algo:EM}. The {\sc em} algorithm converges to a local minimum or saddle point solution.
We refer the reader to~\cite{dempsterjrss77} for more details.

\begin{algorithm}[h!]
\mytopcaption{The {\sc em} algorithm for parameter estimation by likelihood maximization.}
\label{algo:EM}
\begin{algorithmic}[1]
\INPUT ${\cal D} = \{({\bf x}_1,{\bf y}_1),\cdots,({\bf x}_n,{\bf y}_n)\}$, ${\bf w}_0$, $\epsilon$.
\STATE $t \leftarrow 0$
\REPEAT
\STATE Obtain the expectation of objective~(\ref{eq:maximumLikelihood}) under the distribution
$\Pr({\bf h}_i | {\bf x}_i, {\bf y}_i; {\bf w}_t)$.
\STATE Update ${\bf w}_{t+1}$ by maximizing the expectation of objective~(\ref{eq:maximumLikelihood}). Specifically, \\
${\bf w}_{t+1} = \argmax_{\bf w}\sum_i \Pr({\bf h}_i | {\bf x}_i, {\bf y}_i; {\bf w}_t)
\log \Pr({\bf x}_i, {\bf y}_i, {\bf h}_i; {\bf w})$.
\STATE $t \leftarrow t + 1$.
\UNTIL{Objective function cannot be increased above tolerance $\epsilon$.}
\end{algorithmic}
\end{algorithm}


\myparagraph{CCCP Algorithm for Risk Minimization.}

Given the true output ${\bf y}$, we denote the user-specified risk of
predicting $\hat{\bf y}({\bf w})$ as $\Delta({\bf y},\hat{\bf y}({\bf w}))$.
The risk is usually highly non-convex in {\bf w}, and therefore very difficult
to minimize. An efficient way to overcome this difficulty is to use the recently
proposed latent structural support vector machine (hereby referred to as latent
{\sc ssvm}) formulation~\cite{felzenszwalbcvpr08,yuicml09} that minimizes a
regularized upper bound on the risk. Latent {\sc ssvm} provides a
linear prediction rule of the form
$f_{\bf w}({\bf x}) = \argmax_{{\bf y} \in {\cal Y},{\bf h}\in {\cal H}} {\bf w}^\top \Phi({\bf x},{\bf y},{\bf h})$.
Here, $\Phi({\bf x},{\bf y},{\bf h})$ is the joint feature vector. For instance, in our `car' model learning example,
the joint feature vector can be modeled as the {\sc hog}~\cite{dalalcvpr05}
descriptor extracted using pixels in the bounding box {\bf h}.

The parameters {\bf w} are learned by
solving the following optimization problem:
\begin{eqnarray}
&&\min_{{\bf w},\xi_i \geq 0} \frac{1}{2}||{\bf w}||^2 + \frac{C}{n} \sum_{i=1}^n \xi_i, \nonumber \\
\mbox{s.t. } && \max_{h_i \in {\cal H}} {\bf w}^\top \left(\Phi({\bf x}_i,{\bf y}_i,{\bf h}_i) - 
		\Phi({\bf x}_i,\hat{\bf y}_i,\hat{\bf h}_i) \right)
	 \geq \Delta({\bf y}_i,\hat{\bf y}_i) - \xi_i, \nonumber \\
\label{eq:latentSSVM}
&&\forall \hat{\bf y}_i \in {\cal Y}, \forall \hat{h}_i \in {\cal H}, i=1,\cdots,n.
\end{eqnarray}
For any given {\bf w}, the value of $\xi_i$ can be shown to be an upper bound on
the risk $\Delta({\bf y}_i,\hat{{\bf y}}_i({\bf w}))$ (where $\hat{\bf y}_i({\bf w})$ is
the predicted output given {\bf w}). The risk function can also
depend on $\hat{\bf h}_i({\bf w})$; that is, it can be of the form
$\Delta({\bf y}_i,\hat{\bf y}_i({\bf w}),\hat{\bf h}_i({\bf w}))$. We refer the reader to~\cite{yuicml09} for
more details.

Problem~(\ref{eq:latentSSVM}) can be viewed as minimizing the sum
of a convex and a concave function. This observation leads to a
concave-convex procedure ({\sc cccp})~\cite{yuillenc03} outlined in
Algorithm~\ref{algo:latentSSVM}, which has been shown to converge to a
local minimum or saddle point solution~\cite{sriperumbudurnips09}.
The algorithm has two main steps: (i) imputing the hidden variables
(step 3), which corresponds to approximating the concave function by a
linear upper bound; and (ii) updating the value of the parameter using
the values of the hidden variables.  Note that updating the parameters
requires us to solve a convex {\sc ssvm} learning problem (where the output
${\bf y}_i$ is now concatenated with the hidden variable ${\bf h}_i^*$) for
which several efficient algorithms exist
in the literature~\cite{joachimsml09,taskarnips03,tsochantaridisicml04}.

\begin{algorithm}[h!]
\mytopcaption{The {\sc cccp} algorithm for parameter estimation of latent {\sc ssvm}.}
\label{algo:latentSSVM}
\begin{algorithmic}[1]
\INPUT ${\cal D} = \{({\bf x}_1,{\bf y}_1),\cdots,({\bf x}_n,{\bf y}_n)\}$, ${\bf w}_0$, $\epsilon$.
\STATE $t \leftarrow 0$
\REPEAT
\STATE Update ${\bf h}_i^* = \argmax_{h_i \in {\cal H}} {\bf w}_{t}^\top \Phi({\bf x}_i,{\bf y}_i,{\bf h}_i)$.
\STATE Update ${\bf w}_{t+1}$ by fixing the hidden variables for output ${\bf y}_i$ to ${\bf h}_i^*$ and solving the corresponding
{\sc ssvm} problem. Specifically, \\
${\bf w}_{t+1} = \argmin_{\bf w} \frac{1}{2}||{\bf w}||^2 + \frac{C}{n}\sum_i \max\{0,\Delta({\bf y}_i,\hat{\bf y}_i) +
		{\bf w}^\top (\Phi({\bf x}_i,\hat{\bf y}_i,\hat{\bf h}_i) - \Phi({\bf x}_i,{\bf y}_i,{\bf h}_i^*))\}$.
\STATE $t \leftarrow t + 1$.
\UNTIL{Objective function cannot be decreased below tolerance $\epsilon$.}
\end{algorithmic}
\end{algorithm}

\mysection{Self-Paced Learning for Latent Variable Models}
\label{sec:selfPacedLVM}
Our self-paced learning strategy alleviates the
main difficulty of curriculum learning, namely the lack of a readily
computable measure of the {\em easiness} of a sample. In the context
of a latent variable model, for a given parameter {\bf w}, this easiness
can be defined in two ways: (i) a
sample is easy if we are confident about the value of a hidden
variable; or (ii) a sample is easy if it is easy to predict
its true output. The two
definitions are somewhat related: if we are more certain about the
hidden variable, we may be more certain about the prediction. They are
different in that certainty does not imply correctness, and the hidden
variables may not be directly relevant to what makes the output of
a sample easy to predict. 
We therefore focus on the second definition: easy samples are ones
whose correct output can be predicted easily (its likelihood is
high, or it lies far from the margin).

In the above argument, we have assumed a given {\bf w}. However, in
order to operationalize self-paced learning, we need a strategy for
simultaneously selecting the easy samples and learning the parameter
{\bf w} at each iteration. To this end, we note that the parameter
update involves optimizing an objective function that depends on
{\bf w} (for example, see step 4 of both Algorithms~\ref{algo:EM} and~\ref{algo:latentSSVM}). That is,
\begin{equation}
{\bf w}_{t+1} = \argmin_{{\bf w} \in \mathbb{R}^d} \left(r({\bf w}) + \sum_{i=1}^n f({\bf x}_i,{\bf y}_i; {\bf w})\right),
\label{eq:parameterUpdate}
\end{equation}
where $r(.)$ is a regularization function and $f(.)$ is the negative log-likelihood for {\sc em}
or an upper bound on the risk for latent {\sc ssvm} (or any other criteria for parameter learning).
We now modify
the above optimization problem by introducing binary variables $v_i$ that
indicate whether the $i^{th}$ sample is easy or not. Only easy samples contribute to the objective function.
Formally, at each iteration we solve the following mixed-integer
program:
\begin{equation}
({\bf w}_{t+1}, {\bf v}_{t+1}) = \argmin_{{\bf w} \in \mathbb{R}^d,{\bf v} \in \{0,1\}^n}
\left(r({\bf w}) + \sum_{i=1}^n v_if({\bf x}_i,{\bf y}_i; {\bf w}) -
\frac{1}{K}\sum_{i=1}^n v_i\right).
\label{eq:selfPacedMIP}
\end{equation}
$K$ is a weight that determines the number of samples to be
considered: if $K$ is large, the problem prefers to consider only
``easy'' samples with a small value of $f(.)$ (high likelihood, or far
from the margin). Importantly, however, the samples are tied together
in the objective through the parameter ${\bf w}$. Therefore, no sample
is considered independently easy; rather, a \emph{set} of samples is
easy if a ${\bf w}$ can be fit to it such that the corresponding
values of $f(.)$ are small.  We iteratively decrease the value of $K$
in order to estimate the parameters of a latent variable model via
self-paced learning. As $K$ approaches $0$, more samples are included
until problem~(\ref{eq:selfPacedMIP}) reduces to
problem~(\ref{eq:parameterUpdate}). We thus begin with only a few easy
examples, gradually introducing more until the entire training dataset
is used.

To optimize problem~(\ref{eq:selfPacedMIP}), we note that it can be relaxed such that each variable $v_i$ is allowed to take any value
in the interval $[0,1]$. This relaxation is {\em tight}; that is, for any value
of {\bf w} an optimum value of $v_i$ is either $0$ or $1$ for all samples. 
If $f({\bf x}_i,{\bf y}_i;{\bf w}) < 1/K$ then $v_i = 1$ yields the optimal objective function value. Similarly, if
$f({\bf x}_i,{\bf y}_i;{\bf w}) > 1/K$ then the objective is optimal when $v_i = 0$. 

Relaxing problem~(\ref{eq:selfPacedMIP}) allows us to identify special cases
where the optimum parameter update can be found efficiently. One such special case is when $r(.)$
and $f(.)$ are convex in ${\bf w}$, as in the latent {\sc ssvm} parameter update. In this case, the
relaxation of problem~(\ref{eq:selfPacedMIP}) is a biconvex optimization problem.
Recall that a biconvex problem is one
where the variables {\bf z} can be divided into two sets ${\bf z}_1$
and ${\bf z}_2$ such that for a fixed value of each set, the optimal
value of the other set can be obtained by solving a convex
optimization problem. In our case, the two sets of variables are
{\bf w} and {\bf v}. Biconvex problems have a vast literature, with both global~\cite{floudasjota93}
and local~\cite{bazaraa93} optimization techniques. In this work, we use alternative convex search ({\sc acs})~\cite{bazaraa93}, which alternatively optimizes
{\bf w} and {\bf v} while keeping the other set of variables fixed. We found in our experiments 
that {\sc acs} obtained accurate results.

Even in the general case with non-convex $r(.)$ and/or $f(.)$, we can use the alternative search strategy to efficiently
obtain an approximate solution for problem~(\ref{eq:selfPacedMIP}). Given parameters {\bf w}, we can obtain the optimum
{\bf v} as $v_i = \delta(f({\bf x}_i,{\bf y}_i;{\bf w}) < 1/K)$, where $\delta(.)$ is the indicator function.
For a fixed {\bf v}, problem~(\ref{eq:selfPacedMIP}) has the same form as problem~(\ref{eq:parameterUpdate}). Thus, the
optimization for self-paced learning is as easy (or as difficult) as the original parameter learning algorithm.

\myparagraph{Self-Paced Learning for Latent SSVM.} As an illustrative
example of self-paced learning,
%we consider the task of learning the
%parameters of a latent {\sc ssvm}.
Algorithm~\ref{algo:selfPacedLatentSSVM} outlines the overall
self-paced learning method for latent {\sc ssvm}, which involves
solving a modified version of problem~(\ref{eq:latentSSVM}).  At each
iteration, the weight $K$ is reduced by a factor of $\mu > 1$,
introducing more and more (difficult) samples from one iteration to
the next.  The algorithm converges when it considers all samples but
is unable to decrease the latent {\sc ssvm} objective function value
below the tolerance $\epsilon$.  We note that self-paced learning
provides the same guarantees as {\sc cccp}:\\ {\bf Property:}
Algorithm~\ref{algo:selfPacedLatentSSVM} converges to a local minimum
or saddle point solution of problem~(\ref{eq:latentSSVM}). \\ This
follows from the fact that the last iteration of
Algorithm~\ref{algo:selfPacedLatentSSVM} is
%simply
the original {\sc
cccp} algorithm.

\begin{algorithm}[h!]
\mytopcaption{The self-paced learning algorithm for parameter estimation of latent {\sc ssvm}.}
\label{algo:selfPacedLatentSSVM}
\begin{algorithmic}[1]
\INPUT ${\cal D} = \{({\bf x}_1,{\bf y}_1),\cdots,({\bf x}_n,{\bf y}_n)\}$, ${\bf w}_0$, $K_0$, $\epsilon$.
\STATE $t \leftarrow 0$, $K \leftarrow K_0$.
\REPEAT
\STATE Update $h_i^* = \argmax_{h_i \in {\cal H}} {\bf w}_{t}^\top \Phi({\bf x}_i,{\bf y}_i,{\bf h}_i)$.
\STATE Update ${\bf w}_{t+1}$ by using {\sc acs} to minimize the objective
$\frac{1}{2}||{\bf w}||^2 + \frac{C}{n} \sum_{i=1}^n v_i\xi_i - \frac{1}{K} \sum_{i=1}^n v_i$ subject to the
constraints of problem~(\ref{eq:latentSSVM}) as
well as ${\bf v} \in \{0,1\}^n$.
\STATE $t \leftarrow t + 1$, $K \leftarrow K/\mu$.
\UNTIL{$v_i = 1, \forall i$ and the objective function cannot be decreased below tolerance $\epsilon$.}
\end{algorithmic}
\end{algorithm}

%In this case, we solve a modified version of problem~(\ref{eq:latentSSVM}), as shown in Algorithm~\ref{algo:selfPacedLatentSSVM}.
%\begin{equation}
% \label{eq:latentSSVMsp}
%\end{equation}
%$$\frac{1}{2}||{\bf w}||^2 + \frac{C}{n} \sum_{i=1}^n v_i\xi_i - \frac{1}{K} \sum_{i=1}^n v_i$$
%\label{eq:latentSSVMsp}
%\end{eqnarray}
%% solving the following
%% biconvex problem:
%% \begin{eqnarray}
%% &&\min_{{\bf w}\in \mathbb{R}^d,{\bf v}\in [0,1]^n,\xi_i \geq 0}
%% \frac{1}{2}||{\bf w}||^2 + \frac{C}{n} \sum_{i=1}^n v_i\xi_i - \frac{1}{K} \sum_{i=1}^n v_i, \nonumber \\
%% \mbox{s.t. }  && {\bf v} \in \{0,1\}^n, \nonumber \\
%% &&{\bf w}^\top \left( \Phi({\bf x}_i,{\bf y}_i,{\bf h}_i^*) - \Phi({\bf x}_i,\hat{\bf y}_i,\hat{\bf h}_i) \right) \geq
%% \Delta({\bf y}_i,\hat{\bf y}_i,\hat{\bf h}_i)  - \xi_i,  \nonumber \\
%% \label{eq:latentSSVMUpdate}
%% && \forall \hat{\bf y}_i \in {\cal Y}, \hat{\bf h}_i \in {\cal H}, i=1,\cdots,n,
%% \end{eqnarray}
%% where ${\bf h}_i^*$ is the current imputed value of the hidden variable of the $i^{th}$ sample.

Our algorithm requires an initial parameter ${\bf w}_0$ (similar to {\sc cccp}). 
In our experiments, we obtained an estimate of ${\bf w}_0$ by
initially setting $v_i = 1$ for all samples and running the original
{\sc cccp} algorithm for a fixed, small number of iterations $T_0$. 
As our results indicate, this simple strategy was sufficient to obtain
an accurate set of parameters using self-paced learning. 


\mysection{Experiments}
\label{sec:experiments}
We now demonstrate the efficacy of self-paced learning in the context
of latent {\sc ssvm}. We show that our approach outperforms the
state of the art {\sc cccp} algorithm on four standard machine learning
applications. In all our
experiments, the initial weight $K_0$ is set such that the
first iteration selects more than half the samples (as there are
typically more easy samples than difficult ones).
The weight is reduced by a factor $\mu = 1.3$ at each iteration
and the parameters are initialized using $T_0 = 2$ iterations of the original
{\sc cccp} algorithm.

\mysubsection{Noun Phrase Coreference}
\label{subsec:nounphrase}

\paragraph{Problem Formulation.}
Given the occurrence of all the nouns in a document,
the goal of noun phrase coreference is to provide a clustering of the nouns such that each cluster
refers to a single object. This task was formulated within the {\sc ssvm} framework in~\cite{finleyicml05}
and extended to include latent variables in~\cite{yuicml09}. Formally, the input vector {\bf x}
consists of the pairwise features ${\bf x}_{ij}$ suggested in~\cite{ngacl02}
between all pairs of noun phrases $i$ and $j$ in the document.
The output ${\bf y}$ represents a clustering of the nouns. A hidden variable {\bf h} specifies a
forest over the nouns such that each tree in the forest consists of all the nouns of one cluster.
Imputing the hidden variables involves finding the maximum spanning
forest (which can be solved by Kruskal or Prims algorithm).
Similar to~\cite{yuicml09}, we employ two different loss functions, corresponding to the 
pairwise and {\sc mitre} scores.

\myparagraph{Dataset.} We use the publicly available MUC6 noun phrase coreference dataset, which consists
of 60 documents. We use the same split of 30 training and 30 test documents as~\cite{yuicml09}.

\myparagraph{Results.}
\begin{figure}
\threefigurescaps{figures/np.relobj.eps}{figures/np.trainloss.eps}{figures/np.testloss.eps}{3.25cm}{a}{b}{c}
\mycaption{\it Results for the noun phrase coreference experiment. Top: {\sc mitre} score. Bottom:
Pairwise score.
(a) The relative objective value computed as
$(obj_{cccp}-obj_{spl})/obj_{cccp}$, where $obj_{cccp}$ and $obj_{spl}$ are the objective values of
{\sc cccp} and self-paced learning respectively. A green circle indicates a significant improvement (greater than tolerance $C\epsilon$),
while a red circle indicates a significant decline. The black dashed line demarcates equal
objective values. (b) Loss over the training data. Minimum {\sc mitre} loss: $14.48$ and $14.02$ for {\sc cccp} and
self-paced learning respectively;
Minimum pairwise loss: $31.10$ and $31.03$. (c) Loss over the test data. Minimum {\sc mitre} loss: $15.38$ and $14.91$;
Minimum pairwise loss: $34.10$ and $33.93$.
}
\label{fig:nounphrase}
\end{figure}
We tested {\sc cccp} and our self-paced learning method on different
values of $C$;  the average training times over all $40$ experiments
($20$ different values of $C$ and two different loss functions) for
the two methods were $1183$ and $1080$ seconds respectively. 
Fig.~\ref{fig:nounphrase} compares the two methods in terms of the
value of the objective function (which is the main focus of this
work), the loss over the training data and the loss over the test
data. Note that self-paced learning significantly improves the
objective function value in $11$ of the $40$ experiments (compared to
only once when {\sc cccp} outperforms self-paced
learning; see Fig.~\ref{fig:nounphrase}(a)). 
It also provides a better training and testing loss for both {\sc mitre}
and pairwise scores when using the optimal value of $C$ (see Fig.~\ref{fig:nounphrase}(b)-(c)).

\mysubsection{Motif Finding}
\label{subsec:motif}

\paragraph{Problem Formulation.}
We consider the problem of binary classification of {\sc dna} sequences, which was cast as a latent {\sc ssvm} 
in~\cite{yuicml09}. Specifically, the input vector {\bf x} consists of a {\sc dna} sequence of length
$l$ (where each element of the sequence is a nucleotide of type A, G, T or C) and the output space ${\cal Y} = \{+1,-1\}$.
In our experiments, the classes correspond to two different types of
genes: those that bind to a protein of interest with high
affinity and those that do not. The positive sequences are assumed to
contain particular patterns, called \emph{motifs}, of length $m$ that
are believed to be useful for classification. However, the starting
position of the motif within a gene sequence is often not known.
Hence, this position is treated as the hidden variable
${\bf h}$. For this problem, we use the joint feature vector suggested
by~\cite{yuicml09}.
Here, imputing the hidden variables simply involves a search
for the starting position of the motif. The loss function $\Delta$ is the
standard 0-1 classification loss.

\myparagraph{Dataset.} We use the publicly available UniProbe dataset~\cite{bergercell08} that provides positive
and negative {\sc dna} sequences for $177$ proteins. For this work, we chose five proteins at random. The total
number of sequences per protein is roughly $40,000$.
For all the sequences, the motif length $m$ is known (provided with the UniProbe dataset)
and the background Markov model is assumed to be of order $k=3$.
In order to specify a classification task for a particular protein, we randomly split the sequences into roughly 50\% 
for training and 50\% for testing.
\begin{table}[h!]
\small
\begin{center}
(a) Objective function value \\
\begin{tabular}{|c|c|c|c|c|c|}
\hline
{\sc cccp} & $92.77 \pm 0.99$ & ${\bf 106.50 \pm 0.38}$ & $94.00 \pm 0.53$ & $116.63 \pm 18.78$ & $75.51 \pm 1.97$ \\ 
{\sc spl} & ${\bf 92.37 \pm 0.65}$ & $106.60 \pm 0.30$ & ${\bf 93.51 \pm 0.29}$ & ${\bf 107.18 \pm 1.48}$ & ${\bf 74.23 \pm 0.59}$ \\
\hline
\end{tabular} 
\vspace{1mm}
\\(b) Training error (\%) \\
\begin{tabular}{|c|c|c|c|c|c|}
\hline
{\sc cccp} & $27.10 \pm 0.44$ & ${\bf 32.03 \pm 0.31}$ & $26.90 \pm 0.28$ & $34.89 \pm 8.53$ & $20.09 \pm 0.81$ \\
{\sc spl} & ${\bf 26.94 \pm 0.26}$ & $32.04 \pm 0.23$ & ${\bf 26.81 \pm 0.19}$ & ${\bf 30.31 \pm 1.14}$ & ${\bf 19.52 \pm 0.34}$ \\
\hline
\end{tabular}
\vspace{1mm}
\\(c) Test error (\%) \\
\begin{tabular}{|c|c|c|c|c|c|}
\hline
{\sc cccp} & $27.10 \pm 0.36$ & ${\bf 32.15 \pm 0.31}$ & $27.10 \pm 0.37$ & $35.42 \pm 8.19$ & $20.25 \pm 0.65$ \\
{\sc spl} & ${\bf 27.08 \pm 0.38}$ & $32.24 \pm 0.25$ & ${\bf 27.03 \pm 0.13}$ & ${\bf 30.84 \pm 1.38}$ & ${\bf 19.65 \pm 0.39}$ \\
\hline
\end{tabular}
\end{center}
\mycaption{\it Mean and standard deviations for the motif finding experiments using the original {\sc cccp} algorithm (top row) and the proposed
self-paced learning approach (bottom row). The better mean value is highlighted in bold. Note that self-paced learning
provides an improved objective value (the primary concern of this work)
for all proteins. The
improvement in objective value also translates to an improvement in training and test errors.}
\label{table:motif}
\end{table}


\myparagraph{Results.} 
We used five different folds for each protein, randomly initializing
the motif positions for all training samples using four different seed
values (fixed for both methods). We report results for each method
using the best seed (chosen according to the value of the objective
function). For all experiments we use $C=150$ and $\epsilon=0.001$
(the large size of the dataset made cross-validation highly time
consuming). The average time over all $100$ runs for {\sc cccp} and
self-paced learning are $824$ and $1287$ seconds respectively. 
Although our approach is slower than {\sc cccp} for this application, as 
table~\ref{table:motif} shows, it learns a better set of parameters.
While improvements for most folds are small, for the fourth protein,
{\sc cccp} gets stuck in a bad local minimum despite using  multiple
random initializations (this is indicated by the large mean and
standard deviation values).  This behavior is to be expected: in many
cases, the objective function landscape is such that {\sc cccp} avoids
local optima; but in some cases, {\sc cccp} gets stuck in poor local
optimum.  Indeed, over all the $100$ runs ($5$ proteins, $5$ folds and
$4$ seed values) {\sc cccp} got stuck in a bad local minimum $18$
times (where a bad local minimum is one that gave $50\%$ test error)
compared to $1$ run where self-paced learning got stuck. 

Fig.~\ref{fig:motif} shows the average Hamming distance between the
motifs of the selected samples at each iteration of the self-paced learning algorithm.
Note that initially the algorithm selects samples whose motifs have
a low Hamming distance (which intuitively correspond to the easy samples for this application).
It then gradually introduces more difficult samples (as indicated by the rise in the average Hamming
distance). Finally, it considers all samples and attempts to find the most discriminative motif across the entire dataset.
Note that the motifs found over the entire dataset using self-paced
learning provide a smaller average Hamming distance than those found
using the original {\sc cccp} algorithm, indicating a greater
coherence for the resulting output.

\begin{figure}
\threefigures{figures/motif_005_3.hamming.eps}{figures/motif_001_5.hamming.eps}{figures/motif_002_2.hamming.eps}{3.25cm}
\mycaption{\it Average Hamming distance between the motifs found in all selected samples at each iteration. 
Our approach starts with easy samples (small Hamming distance) and gradually introduces more difficult samples (large
Hamming distance) until it starts to consider all samples of the training set. The figure shows results for three
different protein-fold pairs. The average Hamming distance (over all proteins and folds) of the motifs obtained at convergence
are $0.6155$ for {\sc cccp} and $0.6099$ for self-paced learning.}
\label{fig:motif}
\end{figure}

\mysubsection{Handwritten Digit Recognition}
\label{subsec:digits}

\paragraph{Problem Formulation.}
Handwritten digit recognition is a special case of multi-label classification, and hence
can be formulated within the {\sc ssvm} framework.
Specifically, given an input vector {\bf x}, which consists of $m$ grayscale values
that represent an image of a handwritten digit, our aim is to predict the digit.
In other words, ${\cal Y} = \{0,1,\cdots,9\}$. It is well-known that the accuracy
of digit recognition can be greatly improved by explicitly modeling the deformations
present in each image, for example see~\cite{simardnips91}. For simplicity, we assume that
the deformations are restricted to an arbitrary rotation of the image, where the 
angle of rotation is not known beforehand. This angle (which takes a value from
a finite discrete set) is modeled as the hidden variable
{\bf h}. We specify the joint feature vector as
$\Phi({\bf x},{\bf y},{\bf h}) = ({\bf 0}_{{\bf y}(m+1)};\theta_{\bf
  h}({\bf x})\ 1;{\bf 0}_{(9-{\bf y})(m+1)})$, 
where $\theta_{\bf h}({\bf x})$ is the vector representation of the
image {\bf x} rotated by 
the angle corresponding to ${\bf h}$.
In other words, the joint feature vector is the rotated image of the digit
which is padded in the front and back with the appropriate number of
zeroes. Imputing the hidden variables simply involves a search over a
discrete set of angles.  Similar to the motif finding experiment, we
use the standard 0-1 classification loss.

\myparagraph{Dataset.}

We use the standard {\sc mnist} dataset~\cite{lecunieee98}, which
represents each handwritten digit as a vector of length 784 (that is,
an image of size $28\times 28$). For efficiency, we use {\sc pca} to
reduce the dimensionality of each sample to $10$. We perform binary
classification on four difficult digit pairs ($1$-$7$, $2$-$7$,
$3$-$8$, and $8$-$9$), as in \cite{zhangicml07}. The training standard
dataset size for each digit ranges from $5,851$ to $6,742$, and the
test sets range from $974$ to $1,135$ digits. The rotation modeled by
the hidden variable can take one of $11$ discrete values, evenly
spaced between $-60$ and $60$ degrees.

\myparagraph{Results.}

For each digit pair, we use $C$ values ranging from $25$ to $300$, set
$\epsilon=0.001$, and set $K = \frac{10^4}{C}$. Modeling rotation as a
hidden variable significantly improves classification performance,
allowing the images to be better aligned with each other. Across all
experiments for both learning methods, using hidden variables achieves
better test error; the improvement over using no hidden variables is
12\%, 8\%, 11\%, and 22\%, respectively, for the four digit
pairs. {\sc cccp} learning took an average of $18$ minutes across all
runs, while self-paced learning took an average of $53$ minutes.

\begin{figure}[t]
  \begin{tabular}{cccc}
    \hspace{-0.1in}
    \epsfig{file=figures/zero.PC.10.N0.NC2.I17.eps,width=0.23\columnwidth} &
    \epsfig{file=figures/zero.PC.10.N0.NC2.I27.eps,width=0.23\columnwidth} &
    \epsfig{file=figures/zero.PC.10.N0.NC2.I38.eps,width=0.23\columnwidth} &
    \epsfig{file=figures/zero.PC.10.N0.NC2.I89.eps,width=0.23\columnwidth}
  \end{tabular}
  \mycaption{Four digit pairs from {\sc mnist}: $1$-$7$, $2$-$7$, $3$-$8$, $8$-$9$.  Relative objective
        is computed as in Fig.~\ref{fig:nounphrase}.  Positive values
        indicate superior results for self-paced learning. The dotted
        black lines delineate where the difference is greater than the
        convergence criteria range ($C\epsilon$); differences outside
        this range are highlighted in blue.}
\label{fig:mnist}
\end{figure}
%% \begin{figure}[th]
%%   \begin{tabular}{cc}
%%     \begin{minipage}{0.6\columnwidth}
%%       \begin{tabular}{cc}
%%         \epsfig{file=figures/zero.PC.10.N0.NC2.I17.eps,width=0.5\columnwidth} &
%%         \epsfig{file=figures/zero.PC.10.N0.NC2.I27.eps,width=0.5\columnwidth} \\
%%         \epsfig{file=figures/zero.PC.10.N0.NC2.I38.eps,width=0.5\columnwidth} &
%%         \epsfig{file=figures/zero.PC.10.N0.NC2.I89.eps,width=0.5\columnwidth}
%%       \end{tabular}
%%     \end{minipage} &
%%     \hspace{0.1in}
%%     \begin{minipage}{0.3\columnwidth}
%%       \mycaption{Four digit pairs from the {\sc mnist}
%%         dataset. Clockwise from the upper left: $1$ vs. $7$, $2$
%%         vs. $7$, $3$ vs. $8$, $8$ vs. $9$.  Relative primal objective
%%         is the difference between the curriculum learning and {\sc
%%         cccp} objective divided by the {\sc cccp} objective. Positive
%%         values indicate superior results for curriculum learning. The
%%         dotted black lines delineate where the difference is greater
%%         than the convergence criteria range ($C\epsilon$), and
%%         differences outside of this range are highlighted in blue.}
%%     \end{minipage}
%% \label{fig:mnist}
%%   \end{tabular}
%% \end{figure}
%% \begin{figure}[th]
%%   \begin{tabular}{c|c}
%%     \begin{minipage}{0.3\columnwidth}
%%       \epsfig{file=figures/zero.PC.10.N0.NC2.I17.eps,width=1.0\columnwidth}
%%     \end{minipage} &
%%     \begin{minipage}{0.6\columnwidth}
%%       \begin{tabular}{c|c}
%%         \epsfig{file=figures/zero.PC.10.N0.NC2.I27.eps,width=0.5\columnwidth} &
%%         \epsfig{file=figures/zero.PC.10.N0.NC2.I38.eps,width=0.5\columnwidth} \\
%%       \end{tabular}
%%     \end{minipage} \\
%%     \begin{minipage}{0.3\columnwidth}
%%       \epsfig{file=figures/zero.PC.10.N0.NC2.I89.eps,width=1.0\columnwidth}
%%     \end{minipage} &
%%     \begin{minipage}{0.65\columnwidth}
%%       \mycaption{Four digit pairs from the {\sc mnist}
%%         dataset. Relative primal objective is computed as in Fig.~\ref{fig:nounphrase}.
%%         Positive values indicate superior results for self-paced
%%         learning. The dotted black lines delineate where the difference is
%%         greater than the convergence criteria range ($C\epsilon$), and
%%         differences outside of this range are highlighted in blue.}
%%     \end{minipage}
%%   \end{tabular}
%%   \label{fig:mnist}
%% \end{figure}

The above figure compares the training and test errors and
objective values between {\sc cccp} and self-paced
learning. Self-paced learning achieves significantly better values in $15$
runs, and is worse in $4$ runs, demonstrating that it
helps find better solutions to the optimization problems. Though 
training and test errors do not necessarily correlate to objective
values, the best test error across $C$ values is better for self-paced
learning for one of the digit pairs ($1$-$7$), and is the same for
the others.

\mysubsection{Object Localization}
\label{subsec:objectLocalize}

\paragraph{Problem Formulation.} Given a set of images along with
labels that indicate the presence of a particular object
category in the image (for example, a mammal), 
our goal is to learn discriminative object models for all object categories (that is, models
that can distinguish between one object, say bison, from another, say
elephant). In practice, although it is easy to mine such images from
free photo-sharing websites such as Flickr, it is burdensome to obtain
ground truth annotations of the exact location of the object in each
image. To avoid requiring these human annotations, we model
the location of objects as hidden variables. Formally, for a given
image {\bf x}, category {\bf y} and location {\bf h}, the score 
is modelled as ${\bf w}^T\Phi({\bf x},{\bf y},{\bf h}) = {\bf w}_{\bf
  y}^T \Phi_{\bf h}({\bf x})$, where ${\bf w}_{\bf y}$ are the
parameters that corresponds to the class {\bf y} and $\Phi_{\bf
  h}(\cdot)$ is the {\sc hog}~\cite{dalalcvpr05,felzenszwalbcvpr08}
feature extracted from the image at position {\bf h} (the size of 
the object is assumed to be the same for all images --- a reasonable assumption for our datasets).
For the above problem, imputing the hidden variables involves a simple
search over possible locations in a given image.  The loss function
$\Delta({\bf y},\hat{\bf y})$ is again the standard 0-1 classification loss.

\myparagraph{Dataset.} We use images of $6$ different mammals (approximately $45$ images per mammal) that have been
previously employed for object localization~\cite{heitzijcv09}. We split the images of each category into approximately
90\% for training and 10\% for testing.

\myparagraph{Results.} We use five different folds to compare our method with the state of the art
{\sc cccp} algorithm. For each fold, we randomly initialized the location of the object in each image (the
initialization was the same for the two methods). We used a value of $C = 10$ and $\epsilon = 0.001$.
The average training time over all folds were $362$ seconds and
$482$ seconds for {\sc cccp} and self-paced learning respectively. Table~\ref{table:objectLocalize} shows the
mean and standard deviation of three terms: the objective value, the training loss and the testing loss.
Self-paced learning provided a significantly lower (more than tolerance) objective value
than {\sc cccp} for all folds. The better objective value resulted in a substantial improvement in the training
(for $4$ folds) and testing loss (an improvement of approximately 4\%
for achieved for 2 folds). In these experiments, {\sc cccp} never
outperformed self-paced learning for any of the three measures of
performance. 
\begin{table}[h!]
\small
\begin{center}
\begin{tabular}{|c|c|c|}
\hline
Objective & Train Loss (\%) & Test Loss (\%) \\
4.70 $\pm$ 0.11 & 0.33 $\pm$ 0.18  & 16.92 $\pm$ 5.16 \\
\hline
\end{tabular} 
\begin{tabular}{|c|c|c|}
\hline
Objective & Train Loss (\%) & Test Loss (\%) \\
{\bf 4.53 $\pm$ 0.15} & {\bf 0.0 $\pm$ 0.0}  & {\bf 15.38 $\pm$ 3.85} \\
\hline
\end{tabular} 
\end{center}
\mycaption{\it Results for the object localization experiment. Left: {\sc cccp}. Right: Self-paced learning. Note that
self-paced learning provides better results for all measures of performance.}
\label{table:objectLocalize}
\end{table}

Fig.~\ref{fig:objectLocalize} shows the imputed bounding boxes for two images during various iterations of the
two algorithms. The proposed self-paced learning algorithm does not use the hard image during the
initial iterations (as indicated by the red bounding box). In contrast, {\sc cccp} considers all images at
each iteration. Note that self-paced learning provides a more accurate bounding box for the hard image at
convergence, thereby illustrating the importance of learning in a meaningful order.
In our experience, this was a typical behavior of the two algorithms.

\begin{figure}
\centerline{\psfig{file=figures/object_localization.eps,width=0.95\textwidth}}
\mycaption{\it The top row shows the imputed bounding boxes of an easy
and a hard image using the {\sc cccp} algorithm over increasing
iterations (left to right). Note that for the hard (deer) image, the bounding
box obtained at convergence does not localize the object
accurately. In contrast, the self-paced learning approach (bottom row)
does not use the hard image during initial iterations (indicated by
the red color of the bounding box). In subsequent iterations, it is
able to impute accurate bounding boxes for both the easy and hard
image.}
\label{fig:objectLocalize}
\end{figure}


\mysection{Discussion}
\label{sec:discussion}
We proposed the self-paced learning regime in the context of
parameter estimation for latent variable models. Our method works by iteratively
solving a biconvex optimization problem that simultaneously
selects easy samples and updates the parameters. 
Using four standard datasets from disparate domains
(natural language processing, computational biology and
computer vision) we showed that our method outperforms the state
of the art approach.

In the current work, we solve the biconvex optimization problem using
an alternate convex search strategy, which only provides us with a
local minimum solution. Although our results indicate that such a
strategy is more accurate than the state of the art, it is worth noting that
the biconvex problem can also be solved using a global optimization
procedure, for example the one described in~\cite{floudasjota93}.
This is a valuable direction for future work. We are also currently
investigating the benefits of self-paced learning on other computer vision
applications, where the ability to handle large and rapidly growing weakly supervised data
is fundamental to the success of the field.

\myparagraph{Acknowledgements.}
This work is supported by NSF under grant IIS 0917151, MURI contract N000140710747, and the Boeing company.

\newpage

\bibliographystyle{plain}
\bibliography{selfPacedLVM}

\end{document} 
